\section{Příklad 5}
% Jako parametr zadejte skupinu (A-H)
\patyZadani{E}

\subsection{Řešení}
\begin{large}
\flushleft
Sestavíme obyčejnou diferenciální rovnici 1. řádu.
\end{large}

\begin{enumerate}
    \item i = U/R = $i_c$ = $i_R$
    \item $(U_R + u_C) - U = 0$
    \item $u'_C = \frac{1}{C} \cdot i_C$
\end{enumerate}

\begin{enumerate}
    \item[a)] $u'_C = \frac{1}{C} \cdot i_C$
    
    \vspace{0.25cm}
    $u'_C = \frac{1}{C} \cdot \frac{U_R}{R}$
    
    \vspace{0.25cm}
    $u'_C = \frac{U_R}{R \cdot C}$
    
    \vspace{0.25cm}
    \item[b)] $u'_C = \frac{U_R}{R \cdot C}$
    
    \vspace{0.25cm}
    $u'_C = \frac{U - u_C}{R \cdot C}$
    
    \vspace{0.25cm}
    $u'_C + \frac{u_C}{R_C} = \frac{U}{R \cdot C}$
\end{enumerate}

\begin{large}
\vspace{1cm} \flushleft
Očekáváné řešení:
\end{large}

\vspace{0.5cm}
$u_C(t) = K(t) \cdot e^{\lambda \cdot t}$

\vspace{0.25cm}
$\lambda = -\frac{1}{R \cdot C}$

\begin{large}
\vspace{1cm} \flushleft
Derivujeme rovnici.
\end{large}

\vspace{0.5cm}
$u_C = K(t) \cdot e^-\frac{t}{R \cdot C}$

\vspace{0.25cm}
$u'_C = K'(t) \cdot e^{-\frac{t}{R \cdot C}} + K(t) \cdot e^{-\frac{t}{R \cdot C}} \cdot (-\frac{1}{R \cdot C})$

\begin{large}
\vspace{1cm} \flushleft
Dosadíme do předešlého vzorce $u'_C + \frac{u_c}{R \cdot C} = \frac{U}{R \cdot C}$.
\end{large}

\vspace{0.5cm}
$K'(t) \cdot e^{-\frac{t}{R \cdot C}} + K(t) \cdot e^{-\frac{t}{R \cdot C}} \cdot (-\frac{1}{R \cdot C}) + (\frac{1}{R \cdot C}) \cdot K(t) \cdot e^{-\frac{t}{R \cdot C}} = \frac{U}{R \cdot C}$

\begin{large}
\vspace{1cm} \flushleft
Upravíme rovnici a dostaneme následující vzorec:
\end{large}

\vspace{0.5cm}
$K'(t) = \frac{U \cdot e^{-\frac{t}{R \cdot C}}}{R \cdot C}$

\begin{large}
\vspace{1cm} \flushleft
Integrujeme vzorec.
\end{large}

\vspace{0.5cm}
$K'(t) = \frac{U}{R \cdot C} \cdot e^{\frac{t}{R \cdot C}}$

\vspace{0.25cm}
$K(t) = \frac{U}{R \cdot C} \cdot \int e^{\frac{t}{R \cdot C}} \cdot dx$

\vspace{0.25cm}
$K(t) = \frac{U}{R \cdot C} \cdot \frac{R \cdot C}{1} \cdot e^{\frac{t}{R \cdot C}} + k$

\vspace{0.25cm}
$K(t) = U \cdot e^{\frac{t}{R \cdot C}} + k$

\begin{large}
\vspace{1cm} \flushleft
Dosadíme do počáteční podmínky, abychom dostali konstantu k.
\end{large}

\vspace{0.5cm}
$u_C(0) = U + k \cdot e^{-\frac{t}{R \cdot C}}$

\vspace{0.25cm}
$u_C(0) = U + k$

\vspace{0.25cm}
$k = u_C(0) - U$

\begin{large}
\vspace{1cm} \flushleft
Dosadíme do očekáváného řešení.
\end{large}

\vspace{0.5cm}
$u_C(t) = (U \cdot e^{\frac{t}{R \cdot C}} + k) \cdot e^{-\frac{t}{R \cdot C}}$

\vspace{0.25cm}
$u_C(t) = U + k \cdot e^{-\frac{t}{R \cdot C}}$

\begin{large}
\vspace{1cm} \flushleft
Pro kontrolu dosadíme hodnotz pro t = 0.
\end{large}

\vspace{0.5cm}
$u_c = U + (u_c(0) - U)$

\vspace{0.25cm}
$11 = 11$

\begin{large}
\vspace{1cm} \flushleft
Zkouška nám vyšla, tím pádem je naše řešení správné.
\end{large}

\subsection{Výsledek}

$u_C(t) = U + k \cdot e^{-\frac{t}{R \cdot C}}$