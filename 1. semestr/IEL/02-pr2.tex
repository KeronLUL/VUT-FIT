\section{Příklad 2}
% Jako parametr zadejte skupinu (A-H)
\druhyZadani{E}

\subsection{Řešení}
\begin{large}
\flushleft
Obvod pro vypocitani $R_i$:
\end{large}

\vspace{0.25cm}
\begin{circuitikz}
\draw
(0,0) to[R=$R_{1}$] (2,0)
(2,0) to[R=$R_{2}$] (4,0)
(4,0) to[R=$R_3$, *-*] (4,-4)
(4,0) to[R=$R_4$, -*] (7,0)
(7,0) to[short, -o] (8.5,0)
(7,0) to[R=$R_5$, *-*] (7,-4)
(0,-4) to[short, -o] (8.5,-4)
(0,0) to[short] (0,-4)
; \end{circuitikz}

\begin{large}
\vspace{0.5cm} \flushleft
Nejdříve sečteme rezistory $R_1$ a $R_2$.
\end{large}

\vspace{0.25cm}
$R_{12} = R_1 + R_2 = 150 + 335 = 485\: \Omega$

\begin{large}
\vspace{1cm} \flushleft
Poté vypočítáme paralelní rezistory $R_{12}$ a $R_3$, které pak sečteme se seriovým rezistorem $R_4$.
\end{large}

\vspace{0.25cm}
$R_{1234} = \frac{R_{12} \cdot R_{3}}{R_{12} + R_{3}} + R_4 = \frac{485 \cdot 625}{485 + 625} + 245 = 518.0856\: \Omega$

\begin{large}
\vspace{1cm} \flushleft
Zbývají pouze 2 paralelní rezistory, které když zjednodušíme dostaneme $R_i$.
\end{large}

\vspace{0.25cm}
$R_i = \frac{R_{1234} \cdot R_5}{R_{1234} + R_5} = \frac{310851.36}{1118.0856} = 278.0211\: \Omega$

\begin{large}
\vspace{1cm} \flushleft
Obvod pro $U_i$:
\end{large}

\vspace{0.25cm}
\begin{circuitikz}
\draw
(0,0) to[R=$R_{1}$] (2,0)
(2,0) to[R=$R_{2}$] (4,0)
(4,0) to[R=$R_3$, *-*] (4,-4)
(4,0) to[R=$R_4$, -*] (7,0)
(7,0) to[R=$R_5$, -*] (7,-4)
(0,-4) to[short] (7,-4)
(7,0) to[short, -o] (9,0)
(7,-4) to[short, -o] (9,-4)
(9, 0) to[open, v=$U_i$] (9, -4)
(0,0) to[dcvsource, v=$U$] (0,-4)
; \end{circuitikz}

\begin{large}
\vspace{0.5cm} \flushleft
Vypočítáme si napětí nad rezistorem $R_5$, protože $U_5$ = $U_i$. Použijeme následující vzorec:
\end{large}

\vspace{0.5cm}
$U_5 = U \cdot \frac{R_3 \cdot R_5}{(R_1 + R_2) \cdot (R_3 + R_4 + R_5) + R_3 \cdot (R_4 + R_5)} = 250 \cdot \frac{375000}{485 \cdot 1470 + 528125} = 75.5394\: \si{\volt}$

\vspace{1cm}
\begin{circuitikz}
\draw
(0,0) to[R=$R_i$] (3,0)
(3,0) to[R=$R_6$, *-*] (3,-3)
(0,-3) to[short] (3,-3)
(0,0) to[dcvsource, v=$U_i$] (0,-3)
; \end{circuitikz}

\begin{large}
\vspace{0.5cm} \flushleft
Jakmile známe $U_i$, můžeme dopočítat $I_{R6}$ pomocí Théveninovy věty.
\end{large}

\vspace{0.5cm}
$I_{R6} = \frac{U_i}{R_i + R_6} = \frac{75.5394}{428.0211} = 0.1765\: \si{\ampere}$

\begin{large}
\vspace{1cm} \flushleft
Jakmile známe $I_{R6}$, snadno dopočítáme nápětí.
\end{large}

\vspace{0.5cm}
$U_{R6} = I_{R6} \cdot R_6 = 0.1765 \cdot 150 = 26.475\: \si{\volt}$

\subsection{Výsledek}

$U_{R6} = 26.48\: \si{\volt}, I_{R6} = 0.18\: \si{\ampere}$