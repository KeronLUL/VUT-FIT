\section{Příklad 4}
% Jako parametr zadejte skupinu (A-H)
\ctvrtyZadani{A}

\subsection{Řešení}

\begin{circuitikz}
\draw
(0,0) to[vsourcesin, v=$U_{1}$] (3,0)
(3,0) to[cute inductor=$L_{2}$] (5,0)
(5,0) to[short] (6,0)
(6,0) to[short, -*] (6,-2)
(0,0) to[short, -*] (0,-2)
(0,-2) to[short, -*] (3,-2)
(3,-2) to[C=$C_2$] (6,-2)
(3,-2) to[vsourcesin, v=$U_{1}$] (3,-4)
(3,-4) to[cute inductor=$L_{2}$] (6,-4)
(0,-2) to[R=$R_1$] (0,-4)
(0,-4) to[C=$C_1$] (3,-4)
(6,-2) to[R=$R_2$] (6,-4)


; 
\draw [thick, <-] (2,-2.75) arc (0:300:0.3);
\draw (1.15,-2.75) node[]{$I_B$};
\draw [thick, <-] (3,-1) arc (0:300:0.3);
\draw (2.15,-1) node[]{$I_A$};
\draw [thick, <-] (5.2,-2.75) arc (0:300:0.3);
\draw (4.3,-2.75) node[]{$I_C$};
\end{circuitikz}

\begin{large}
\flushleft
Vypočítáme si $\omega$.
\end{large}

\vspace{0.5cm}
$\omega = 2\pi \cdot f = 2\pi \cdot 70 = 439.8230$

\begin{large}
\vspace{1cm}\flushleft
Vypočítáme si impedanci cívek a kondenzádorů.
\end{large}

\vspace{0.5cm}
$Z_{C1} = -\frac{j}{\omega \cdot C_1} = -\frac{j}{439.8230 \cdot 2 \cdot 10^{-4}} = -\frac{j}{0.0880} = -j \cdot 11.3636\: \Omega$

\vspace{0.25cm}
$Z_{C2} = -\frac{j}{\omega \cdot C_2} = -\frac{j}{439.8230 \cdot 1.05 \cdot 10^{-4}} = -\frac{j}{0.0462} = -j \cdot 21.6450\: \Omega$

\vspace{0.25cm}
$Z_{L1} = j\omega L_1 = j \cdot 52.7788\: \Omega$

\vspace{0.25cm}
$Z_{L2} = j\omega L_2 = j \cdot 43.9823\: \Omega$

\begin{large}
\vspace{1cm} \flushleft
Napišeme si rovnice pro smyčkové proudy.
\end{large}

\vspace{0.5cm}
A: $Z_{L1} \cdot I_A + Z_{C2} \cdot I_A - Z_{C2} \cdot I_C + u_1 = 0$

\vspace{0.25cm}
B: $Z_{C1} \cdot I_B + R_1 \cdot I_B + u_2 = 0$

\vspace{0.25cm}
C: $R_2 \cdot I_C + Z_{L2} \cdot I_C + Z_{c2} \cdot I_C - Z_{C2} \cdot I_A - u_2 = 0$

\begin{large}
\vspace{1cm} \flushleft
Přepíšu si rovnice na soustavu rovnic.
\end{large}

\vspace{0.5cm}
$
\begin{array}{ccccccc}
    I_A \cdot (Z_{L1} + Z_{C2}) &  &  & - & I_C \cdot Z_{C2} & = & -u_1\\
    &  & I_B \cdot (Z_{C1} + R_1) &  &  & = & -u_2\\
    -I_A \cdot Z_{C2} & & & + & I_C \cdot (R_2 + Z_{L2} + Z_{C2}) & = & u_2\\
\end{array}
$

\begin{large}
\vspace{1cm} \flushleft
Přepíšeme soustavu do matice, dosadíme a vypočítáme.
\end{large}

\vspace{0.5cm}
$
\left(
\begin{array}{ccc|c}
    I_A \cdot (Z_{L1} + Z_{C2}) & 0 & -I_C \cdot Z_{C2} & -u_1\\
    0 & I_B \cdot (Z_{C1} + R_1) & 0 & -u_2\\
    -I_A \cdot Z_{C2} & 0 & I_C \cdot (R_2 + Z_{L2} + Z_{C2}) & u_2\\
\end{array}
\right)
$

\vspace{0.5cm}
$
\left(
\begin{array}{ccc|c}
    0 + 31.1338j & 0 & 0 + 21.6450j & -35\\
    0 & 12 - 11.3636j & 0 & -55\\
    0 + 21.6450j & 0 & 14 + 22.3373j & 55\\
\end{array}
\right)
\; \sim
$

\vspace{0.5cm}
$
\left(
\begin{array}{ccc|c}
    1 & 0 & 0.6952j & 1.1242\\
    0 & 12 - 11.3636j & 0 & -55\\
    0 & 0 & 14 + 7.2892j & 79.3329\\
\end{array}
\right)
\; \sim
$

\vspace{0.5cm}
$
\left(
\begin{array}{ccc|c}
    1 & 0 & 0.6952j & 1.1242\\
    0 & 1 & 0 & -2.4164 - 2.2883j\\
    0 & 0 & 14 + 7.2892j & 79.3329\\
\end{array}
\right)
$
\: \: $\sim$ \: \:
$
\left(
\begin{array}{ccc|c}
    1 & 0 & 0 & -3.0994 + 2.7379j\\
    0 & 1 & 0 & -2.4164 - 2.2883j\\
    0 & 0 & 1 & 4.4581 - 2.3211j\\
\end{array}
\right)
$

\vspace{0.5cm}
$I_A = -3.0994 + 2.7379j$

\vspace{0.25cm}
$I_B = -2.4164 - 2.2883j$

\vspace{0.25cm}
$I_C = 4.4581 - 2.3211j$

\begin{large}
\vspace{1cm} \flushleft
Dostaneme smyčkové proudy. Vypočítáme pomocí nich $I_{C2}$.
\end{large}

\vspace{0.5cm}

$I_{C2} = I_C - I_A = (4.4581 - 2.3211j) - (-3.0994 + 2.7379j) = 7.5575 - 5.059j$

\vspace{0.25cm}
$|I_{C2}| = \sqrt{A^2 + B^2} = \sqrt{(7.5575)^2 + (-5.059)^2} = 9.0945\: \si{\ampere}$

\begin{large}
\vspace{1cm} \flushleft
Pomocí $I_{C2}$ a impedance $Z_{C2}$ vypočítáme $U_{c2}$.
\end{large}

\vspace{0.5cm}
$U_{c2} = I_{C2} \cdot Z_{C2} = (7.5575 - 5.059j) \cdot (-21.6450j) = -109.5021 - 163.5821j$

\vspace{0.25cm}
$|U_{C2}| = \sqrt{(A)^2 + (B)^2} = \sqrt{(-109.5021)^2 + (-163.5821)^2} = 196.8497\: \si{\volt}$

\begin{large}
\vspace{1cm} \flushleft
Zbývá vypočítat fázový posun pomocí vzorce $\varphi = \arctan (\frac{B}{A})$, kde A je reálná část $I_{C2}$ a B je imaginární část $I_{C2}$. Nacházíme se ve 4. kvadrantu proto přičteme $\pi$.
\end{large}

\vspace{0.5cm}
$\varphi = \arctan (\frac{B}{A}) + \pi = \arctan (\frac{-5.059}{7.5575}) + \pi = 2.5517\: \si{\radian}$

\subsection{Výsledek}

$|U_{C2}| = 196.85\: \si{\volt}, \varphi = 2.55\: \si{\radian}$