\section{Příklad 3}
% Jako parametr zadejte skupinu (A-H)
\tretiZadani{H}

\subsection{Řešení}

\begin{circuitikz} \draw
(2,4) to[R,l=$R_1$, v=$U_A$] (2,0)--(0,0)
(5.5,4) to[R, l=$R_4$, v=$U_{R4}$, i=$I_{R4}$] (5.5,0) to[R, l=$R_3$, v=$U_C$ ,*-*] (2,0)
(0,0) to[I, i=$I_1$] (0,4)
(0,4) to[short] (2,4)
(2,4) to[R=$R_2$, *-*] (5.5,4)
(5.5,4) to[R=$R_5$, *-] (8,4)
(8,4) to[dcvsource, v=$U$] (8,0)
(8,0) to[short] (5.5,0)
(2,0) to[short] (2,-1.5)
(2,-1.5) to[I, i=$I_2$] (5.5,-1.5)
(5.5,0) to[short] (5.5,-1.5)
(5.5,4) to[open, v=$U_B$] (2,0)
(1.75,-0.25) node{R}
(5.74,-0.25) node{C}
(1.75,4.25) node{A}
(5.75, 4.25) node{B}
;
\end{circuitikz}


\begin{large}
Napišeme rovnice podle 1. Kirchhoffova zákona.
\end{large}

\vspace{0.5cm}
$A: I_1 + I_{R2} - I_{R1} = 0$

\vspace{0.25cm}
$B: I_{R5} - I_{R2} - I_{R4} = 0$

\vspace{0.25cm}
$C: I_2 + I_{R4} - I_{R5} - I_{R3} = 0$

\begin{large}
\vspace{1cm} \flushleft
Dále si napíšeme rovnice pro každý proud, abychom mohli dosadit do soustavy rovnic a vypočítat neznámé.
\end{large}

\vspace{0.5cm}
$I_{R1} = \frac{U_A}{R_1}$

\vspace{0.25cm}
$I_{R2} = \frac{U_B - U_A}{R_2}$

\vspace{0.25cm}
$I_{R3} = \frac{U_C}{R_3}$

\vspace{0.25cm}
$I_{R4} = \frac{U_B - U_C}{R_4}$

\vspace{0.25cm}
$I_{R5} = \frac{U + U_C - U_B}{R_5}$

\begin{large}
\vspace{1cm} \flushleft
Dosadíme do soustavy Kirchhoffova zákonu.
\end{large}

\vspace{0.5cm}
$0.95 + \frac{U_B + U_A}{39} - \frac{U_A}{47} = 0$

\vspace{0.25cm}
$\frac{130 + U_C - U_B}{25} - \frac{U_B - U_A}{39} - \frac{U_B - U_C}{28} = 0$

\vspace{0.25cm}
$0.50 + \frac{U_B - U_C}{28} - \frac{130 + U_C - U_B}{25} - \frac{U_C}{58} = 0$

\begin{large}
\vspace{1cm} \flushleft
Upravíme soustavu.
\end{large}

\vspace{0.5cm}
$
\begin{array}{ccccccc}
    -86U_A & + & 47U_B & & & = & -1741.35\\
    700U_A & - & 2767U_B & + & 2067U_C & = & -141960\\
    & & 3074U_B & - & 3774U_C & = & 170520\\
\end{array}
$

\begin{large}
\vspace{1cm} \flushleft
Přepíšeme soustavu do matice a vypočítáme.
\end{large}

\vspace{0.5cm}
$
\left(
\begin{array}{ccc|c}
-86 & 47 &0 & -1741.35\\
700 & -2767 & 2067 & -141960\\
0 & 3074 & -3774 & 170520\\
\end{array}
\right)
$
\: \: $\sim$ \: \:
$
\left(
\begin{array}{ccc|c}
1 & -0.5465 & 0 & 20.2483\\
0 & -2384.4419 & 2067 & -156133.7791\\
0 & 3074 & -3774 & 170520\\
\end{array}
\right)
$
\: \: $\sim$ \: \:

\vspace{0.5cm}
$
\left(
\begin{array}{ccc|c}
1 & 0 & -0.4736 & 56.034\\
0 & 1 & -0.8669 & 65.4802\\
0 & 0 & -1109.2431 & -30766.1982\\
\end{array}
\right)
$
\: \: $\sim$ \: \:
$
\left(
\begin{array}{ccc|c}
1 & 0 & 0 & 69.1741\\
0 & 1 & 0 & 89.5239\\
0 & 0 & 1 & 27.7362\\
\end{array}
\right)
$

\begin{large}
\vspace{1cm} \flushleft
Z matice dostaneme hodnotu $U_C$ a díy ní dopočítáme $U_A$ a $U_B$. Vyjde nám následující:
\end{large}

\vspace{0.5cm}
$U_A = 69.1741\: \si{\volt}$

\vspace{0.25cm}
$U_B = 89.5239\: \si{\volt}$

\vspace{0.25cm}
$U_C = 27.7362\: \si{\volt}$

\begin{large}
\vspace{1cm} \flushleft
Dosadíme do vzorce pro $I_{R4}$ a vypočítáme.
\end{large}

\vspace{0.5cm}
$I_{R4} = \frac{U_B - U_C}{R_4} = \frac{61.7877}{28} = 2.2067\: \si{\ampere}$

\begin{large}
\vspace{1cm} \flushleft
Provedeme kontrolu dosazením do rovnice.
\end{large}

\vspace{0.5cm}
$I_{R5} - I_{R2} - I_{R4} = 0$

\vspace{0.25cm}
$2.7285 - 2.2067 - 0.5218 = 0$

\vspace{0.25cm}
$0 = 0$

\begin{large}
\vspace{1cm}
Zkouška nám vyšla, proto můžeme dopočítat $U_{R4}$ pomocí Ohmova zákonu.
\end{large}

\vspace{0.5cm}
$U_{R4} = I_{R4} \cdot R_4 = 2.2067 \cdot 28 = 61.7876\: \si{\volt}$

\subsection{Výsledek}
$I_{R4} = 2.21\: \si{\ampere}, U_{R4} = 61.79\: \si{\volt}$